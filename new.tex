\documentclass[a4paper,10pt]{article}
\usepackage[colorlinks=true]{hyperref}
\usepackage[utf8x]{inputenc}
\usepackage{fontspec}
asd\usepackage{polyglossia}
\usepackage{xltxtra} % standard for nearly all XeLaTeX documents
\usepackage{tikz-dependency}
\usepackage{tikz}
\usepackage{tikz-qtree}

\defaultfontfeatures{Mapping=tex-text} % ditto
\setmainfont{Gentium}
\newfontfamily\d[Script=Devanagari]{Lohit Devanagari}
\usepackage[margin=0.9in]{geometry}
\setdefaultlanguage{english}
\setotherlanguage{hindi}


%\usepackage[pdftex,bookmarks,colorlinks]{hyperref}
%\usepackage{epsfig}
%\usepackage{fixltx2e}


%\newfontfamily\devtransl[Script=Roman,Mapping=DevRom]{Times New Roman}
%\setdefaultlanguage{sanskrit}

%opening
\title{Notes on CPG Class}
\author{}

\begin{document}

\maketitle

\newpage

\tableofcontents

\newpage

\begin{abstract}

  This notes is written for the Independent Study on Computational Paninian Grammar undertaken by us under the guidance of Prof Dipti Misra.
  In this, we explore concepts of the paninian framework along with the its cognitive aspects postulated by Bhartrihari.
  We then proceed to the concept of Vivaksha ( {\d विवक्षा }) and propose a computational model of it.

\end{abstract}
\newpage
\section{Introduction}
We attempt to understand Panini's approach of analyzing language. Panini was a Sanskrit grammarian around 400 BC who wrote a precise grammar, arguably computational in its ways.
The grammar, Ashtadhyayi  defines usage rules for phonology, morphology, case and karakas and inflections in the Sankrit language by way of rules. The rules are scoped with progessively narrower scope in a default-exception manner.
Thus, there are some default rules and after that exceptions are specified.
The basic approach for analyzing structure is a dependency approach.
The grammar defines karakas and vibhaktis linked with each other along with modifiers of karakas to get a structure of the sentence.
The karakas are a syntactico-semantic labels given to chunks in the sentence which are "marked" with vibhaktis.
The complexity arises from the many to many karaka vibhakti mapping, thus we have defaults and exceptions. That is how Panini overcomes it.
\paragraph{} One thing to note is that the grammar does not explain its rules, its a listing which just works. Many scholars have tried to provide explanations of the grammar- Patanjali in Mahabhasya, Bhartrihari in Vakyapadiya,etc. Patanjali points out that since the term karaka is not expanded upon, thus it is to be interpreted literally i.e. karoti iti - one who does, the author of Kashika interprets karaka as "nimitta" - cause of the action. Bhartrihari introduces a new term sadhana which is introduced below.



\newpage
\section{Introduction to Bhartrihari's Approach}
% Introduction to Bhartrihari

Bhartrihari introduces the notion of \textit{``power" ({\d शक्ति.})} to explain various phenomenon. 
According to him, every object is a bundle of powers which correspond to the ability of this object to participate and perform in various actions. 
We can define the object itself by describing its powers. For example, a lamp has its ability/power to illuminate, a pen has an ability to write,
a book has both the ability to read from and write into.

\paragraph{} These powers are dynamic, intertwined and complex. Every power manifests itself through a substratum, through one or many concrete objects. 
These powers, in the form of concrete objects aid as the means in accomplishment of the action. Because the concrete object has the ability/power 
to accomplish the action, it is called ``{\d साधन्} (means)".
In accomplishing the action, the object lends a particular power of its own, and acts as the means. So, Bhartrihari postulates that, it is not really the object that is the means but it is its power.

\paragraph{} Consider the sentence, ``{\d चाबी बॆ ताला खॊला }|" which means ``the keys opened the lock". Here, the action is ``opening". 
There are two concrete objects which are ``the keys" and the ``lock". Following Bhartrihari, the power of the ``keys" is the ``ability to open locks". 
The power of the ``lock" is the ``the ability to get opened" (It also has ablities like ``Guarding", ``Sealing" etc but, in the context, this applies best). 
When we look at the original sentence and try to map it to the real world, we will be faced with a severe problem which is, ``keys are inanimate", 
they cannot open the lock by themselves. But, this sentence applies well in real communication and people use it. When we look at this sentence and say 
instead of the objects themselves, the powers are the means of accomplishment then, the sentence can be at a vague meta level be represented as 
`` an object with the ability to open, opened the object which has the ability to be opened". 

\paragraph{} This creates distinction between what Bhartrihari calls as ``sabdhArtha" (Spoken meaning) and ``vAsthavArtha" (Meaning in the real-world). 
The sentences like above are possible because the speaker of the language is not totally bound by the realities of the world. 
The speakers' cognitive model is based on the powers of the objects and the resulting sentences are the output of his intention and his cognition.

\paragraph{} These ``{\d साधन्} (means)" are embodied into the sentence as ``{\d कारक } roles" which are described in the previous section. 
One can see, in various instances the same object being used as different ``{\d कारक }". For example, we have 
\phantomsection
\hypertarget{fig1}{}

\begin{center}
\begin{dependency}[arc edge, arc angle=80, text only label, label style={above}]
   \begin{deptext}[column sep=1em]
      dEvadatt-ne \& Ag-se \& kaDAi-me \& khAna \& pakAya \\
      dEvadatta [ERG] \& with Fire [INS] \& in the vessel [LOC] \& food [ACC]  \& cooked \\
   \end{deptext}
   \deproot{5}{ROOT}
   \depedge{5}{1}{karta}
   \depedge{5}{2}{karaNa}
   \depedge{5}{3}{adhikaraNa}
   \depedge{5}{4}{karma}
\end{dependency} 

 ``(1) dEvadatta cooked the rice with the fire in the vessel " 
 

\end{center}

\phantomsection
\hypertarget{fig2}{}

\begin{center}
\begin{dependency}[arc edge, arc angle=80, text only label, label style={above}]
   \begin{deptext}[column sep=1em]
      kaDAi-ne \& khAna \& pakAya \\
      the vessel [NOM] \& food [ACC]  \& cooked \\
   \end{deptext}
   \deproot{3}{ROOT}
   \depedge{3}{1}{karta}
   \depedge{3}{2}{karma}
\end{dependency} 

 ``(2) The vessel cooked the rice " 
 

\end{center}

\phantomsection
\hypertarget{fig1}{}

\begin{center}
\begin{dependency}[arc edge, arc angle=80, text only label, label style={above}]
   \begin{deptext}[column sep=1em]
      dEvadatt-ne \& is kaDAi-ko \& banAya \\
      dEvadatta [ERG] \& this vessel [ACC]  \& made \\
   \end{deptext}
   \deproot{3}{ROOT}
   \depedge{3}{1}{karta}
   \depedge{3}{2}{karma}
\end{dependency} 

 ``(3) dEvadatta made this vessel " 
 

\end{center}

\paragraph{} We see that, in the previous three examples, the ``vessel" is used as karta, karma and adhikaraNa. 
Considering the semantic distinction between the three ``{\d कारक } roles", any concrete object cannot probably undergo changes so dramatic that 
it becomes the agent once, the object once and an instrument in an other instant. So, here the speaker while shifting the concrete object 
through all these situations is actually placing the object in the sentence, using it as a substratum for the specific power 
that is there in this object which applies in the given context. 
In the first sentence the, speaker focuses on the ``ability to hold food" hence the adhikaraNa kArakA. 
In the next, he elevates the vessel to a central factor in the act of cooking making it the ``karta".
Finally, in the last sentence, he focuses on its ``ability to be made". 
This phenomenon of depicting various real world entities via different kArakAs in different contexts as the speaker intends and sees fit is called ``vivaksha".

\paragraph{} Bhartrihari states that, for a speaker to produce an utterance, the objects in the sentence need not 
exist in the real world but they merely have to exist as ``objects of cognition" and the sentence is but a manifestation of an ``Internal Reality".

\newpage
\section{karta}
  \subsection{Introduction}
\newpage
\section{karma}
  \subsection{Introduction}
\newpage
\section{karaNa}
  \subsection{Introduction}
\newpage
\section{sampradAnA}
  \subsection{Introduction}
\newpage
\section{apAdAna}
  \subsection{Introduction}
\newpage
\section{adhikaraNa}
  \subsection{Introduction}
\newpage
\section{The other relations}
  \subsection{hEtu}
  \subsection{tAdarthya}
  \subsection{sEsa}

\newpage 
\section{Pratibha}
Now, Let us consider from where the meaning of a sentence or a situation arises. 
We gather knowledge from the world around us and present it in the form of sentences.
We percive the external situation and connect with it using our senses. 
So, we might postulate that meaning of that situation arises from the \textbf{sense-contact / perception}.

\paragraph{} But, sometimes we understand things with out really employing our senses. 
This knowledge comes from \textbf{inference}.
For example, when we see a smoke, we \textit{infer} that there is a fire even when we do not actually see the fire. 
When we see a man dead, and a knife clutched in his hands we infer that the knife may have been the cause of death.

\paragraph{} Another source of knowledge is from a \textbf{reliable person}. 
Depending upon the extent of reliablity of the person we place trust in the knowledge that he has passed on.

\paragraph{} With the above three sources of knowledge, there are some problems. Each person percieves a situation in a different way. 
We are limited by the skills of our perception. The same person may percive something in different ways at different times. 
Seen from a distance, or seen within a confusion, one might percieve no difference between two colors but, closely examined, they might.


\newpage
\section{Vivaksha ?}
\newpage
\section{Modeling Vivaksha}
\end{document}
