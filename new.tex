\documentclass[a4paper,10pt]{article}
\usepackage[utf8x]{inputenc}
\usepackage{fontspec}
\usepackage{polyglossia}
\usepackage{xltxtra} % standard for nearly all XeLaTeX documents
\defaultfontfeatures{Mapping=tex-text} % ditto
\setmainfont{Gentium}
\newfontfamily\d[Script=Devanagari]{Lohit Devanagari}
\usepackage[margin=0.9in]{geometry}
\setdefaultlanguage{english}
\setotherlanguage{hindi}

%\usepackage[pdftex,bookmarks,colorlinks]{hyperref}
%\usepackage{epsfig}
%\usepackage{fixltx2e}


%\newfontfamily\devtransl[Script=Roman,Mapping=DevRom]{Times New Roman}
%\setdefaultlanguage{sanskrit}

%opening
\title{Notes on CPG Class}
\author{}

\begin{document}

\maketitle

\newpage

\tableofcontents

\newpage

\begin{abstract}

  This notes is written for the Independent Study on Computational Paninian Grammar undertaken by us under the guidance of Prof Dipti Misra.
  In this, we explore concepts of the paninian framework along with the its cognitive aspects postulated by Bhartrihari.
  We then proceed to the concept of Vivaksha ( {\d विवक्षा }) and propose a computational model of it.

\end{abstract}
\newpage
\section{Introduction}
{\d मॆरा नाम् ग़न्णॆश् है |} Hello!


\newpage
\section{Introduction to Bhartrihari's Approach}
% Introduction to Bhartrihari


\section{On KArakAs}



\end{document}
