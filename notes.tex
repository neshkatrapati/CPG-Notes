%
% This is the LaTeX template file for lecture notes for CS267,
% Applications of Parallel Computing.  When preparing 
% LaTeX notes for this class, please use this template.
%
% To familiarize yourself with this template, the body contains
% some examples of its use.  Look them over.  Then you can
% run LaTeX on this file.  After you have LaTeXed this file then
% you can look over the result either by printing it out with
% dvips or using xdvi.
%

\documentclass[twoside]{article}
\setlength{\oddsidemargin}{0.25 in}
\setlength{\evensidemargin}{-0.25 in}
\setlength{\topmargin}{-0.6 in}
\setlength{\textwidth}{6.5 in}
\setlength{\textheight}{8.5 in}
\setlength{\headsep}{0.75 in}
\setlength{\parindent}{0 in}
\setlength{\parskip}{0.1 in}
\usepackage{tikz-dependency}

%
% ADD PACKAGES here:
%

\usepackage{amsmath,amsfonts,graphicx}

%
% The following commands set up the lecnum (lecture number)
% counter and make various numbering schemes work relative
% to the lecture number.
%
\newcounter{lecnum}
\renewcommand{\thepage}{\thelecnum-\arabic{page}}
\renewcommand{\thesection}{\thelecnum.\arabic{section}}
\renewcommand{\theequation}{\thelecnum.\arabic{equation}}
\renewcommand{\thefigure}{\thelecnum.\arabic{figure}}
\renewcommand{\thetable}{\thelecnum.\arabic{table}}

%
% The following macro is used to generate the header.
%
\newcommand{\lecture}[4]{
   \pagestyle{myheadings}
   \thispagestyle{plain}
   \newpage
   \setcounter{lecnum}{#1}
   \setcounter{page}{1}
   \noindent
   
   \begin{center}
   \framebox{
      \vbox{\vspace{2mm}
    \hbox to 6.28in { {\bf Independent Study: Computational Paninian Grammar
		\hfill Spring 2013} }
       \vspace{4mm}
       \hbox to 6.28in { {\Large \hfill Lecture #1: #2  \hfill} }
       \vspace{2mm}
       \hbox to 6.28in { {\it Professor: #3 \hfill Tags: #4 } }
      \vspace{2mm}}
   }
   \end{center}
   \markboth{Lecture #1: #2}{Lecture #1: #2}

}
%
% Convention for citations is authors' initials followed by the year.
% For example, to cite a paper by Leighton and Maggs you would type
% \cite{LM89}, and to cite a paper by Strassen you would type \cite{S69}.
% (To avoid bibliography problems, for now we redefine the \cite command.)
% Also commands that create a suitable format for the reference list.
\renewcommand{\cite}[1]{[#1]}
\def\beginrefs{\begin{list}%
        {[\arabic{equation}]}{\usecounter{equation}
         \setlength{\leftmargin}{2.0truecm}\setlength{\labelsep}{0.4truecm}%
         \setlength{\labelwidth}{1.6truecm}}}
\def\endrefs{\end{list}}
\def\bibentry#1{\item[\hbox{[#1]}]}

%Use this command for a figure; it puts a figure in wherever you want it.
%usage: \fig{NUMBER}{SPACE-IN-INCHES}{CAPTION}
\newcommand{\fig}[3]{
			\vspace{#2}
			\begin{center}
			Figure \thelecnum.#1:~#3
			\end{center}
	}
% Use these for theorems, lemmas, proofs, etc.
\newtheorem{theorem}{Theorem}[lecnum]
\newtheorem{lemma}[theorem]{Lemma}
\newtheorem{proposition}[theorem]{Proposition}
\newtheorem{claim}[theorem]{Claim}
\newtheorem{corollary}[theorem]{Corollary}
\newtheorem{definition}[theorem]{Definition}
\newenvironment{proof}{{\bf Proof:}}{\hfill\rule{2mm}{2mm}}

% **** IF YOU WANT TO DEFINE ADDITIONAL MACROS FOR YOURSELF, PUT THEM HERE:

\newcommand\E{\mathbb{E}}

\begin{document}
%FILL IN THE RIGHT INFO.sa
%\lecture{**LECTURE-NUMBER**}{**DATE**}{**LECTURER**}{**SCRIBE**}

\lecture{1}{January 21}{Dr. Dipti Misra}{CPG,Intro}
%\footnotetext{These notes are partially based on those of Nigel Mansell.}

% **** YOUR NOTES GO HERE:
\subsection{Introduction}
The Karaka system serves as the basis for description of Panini's Syntax. It is a syntacito-semantic representation of the relations between the verb and the direct participants of the action in the sentence.
\subsection{Definitions}
Panini's work is explained, extended, commented and reinterpreted by many authors like kAtyayana, patanjali, bhartrhari and others. This section includes some of the definitions of the ``kAraka".

\begin{itemize}
 \item \textbf{Patanjali}, in his Mahabhashya defines ``kAraka" as ``karOti iti" ( ``The one that does" )
 \item The author of \textbf{kAsika} explains it as being synonymous to ``hEtu" and ``nimitta"  (Cause) - ``kArakam hEtur ity anarthAntaram" ( `` Cause and kArakam are one and the same " )
 \item \textbf{Bhartrhari} uses the term ``sAdhanam" to specify kAraka as the one capable of establishing action which is given the term ``sAdhya".
 \item \textbf{NagEsa} defines kAraka as the one that producces the action. 
\end{itemize}

Therefore, we may say that kAraka is a animate/inanimate, passively/actively involved entity in the acomplishment of an action. The relations between the verb ( ``kriya" ) and the kAraka are of the type visheshaNa - visheshya ( Modifier - Modified ).

There are six kArakas. They are specified below briefly. ( Written as per the order )\\ \\
\framebox{\vbox{
  \begin{itemize}
    \item \textbf{apAdAnam}  :  Defined as ``dhruvam apAye pAdAnam" - The Entity which remains constant when seperation takes place
    \item \textbf{sampradAnam}  : ``karmaNA yam abhipraiti sa sampradAnam" - Is the entity for which the karma is intended. 
    \item \textbf{karaNam} :
    \item \textbf{adhikaraNam} 
    \item \textbf{karma} 
    \item \textbf{karta} 
  \end{itemize}
}}
\subsection{Examples}
Consider the following sentence.
\begin{center}
\begin{dependency}[theme = simple]
   \begin{deptext}[column sep=1em]
      dEvadattah \& kAshTaih \& sthAlyAm \& tAnDulAn \& pacati \\
      dEvadatta [NOM] \& with Firewood [INS] \& in the vessel [LOC] \& rice [ACC]  \& cooks \\
   \end{deptext}
   \deproot{5}{ROOT}
   \depedge{5}{1}{karta}
   \depedge{5}{2}{karaNa}
   \depedge{5}{3}{dEshAdhikaraNa}
   \depedge{5}{4}{karma}
\end{dependency} 

 `` dEvadatta cooks the rice with the firewood in the vessel " \\
\end{center}


\section{Some theorems and stuff} % Don't be this informal in your notes!

We now delve right into the proof.

\begin{lemma}
This is the first lemma of the lecture.
\end{lemma}

\begin{proof}
The proof is by induction on $\ldots$.
For fun, we throw in a figure.
%%%NOTE USAGE !
\fig{1}{1in}{A Fun Figure}

This is the end of the proof, which is marked with a little box.
\end{proof}

\subsection{A few items of note}

Here is an itemized list:
\begin{itemize}
\item this is the first item;
\item this is the second item.
\end{itemize}

Here is an enumerated list:
\begin{enumerate}
\item this is the first item;
\item this is the second item.
\end{enumerate}

Here is an exercise:

{\bf Exercise:}  Show that ${\rm P}\ne{\rm NP}$.

Here is how to define things in the proper mathematical style.
Let $f_k$ be the $AND-OR$ function, defined by

\[ f_k(x_1, x_2, \ldots, x_{2^k}) = \left\{ \begin{array}{ll}

	x_1 & \mbox{if $k = 0$;} \\

	AND(f_{k-1}(x_1, \ldots, x_{2^{k-1}}),
	   f_{k-1}(x_{2^{k-1} + 1}, \ldots, x_{2^k}))
	 & \mbox{if $k$ is even;} \\

	OR(f_{k-1}(x_1, \ldots, x_{2^{k-1}}),
	   f_{k-1}(x_{2^{k-1} + 1}, \ldots, x_{2^k}))	
	& \mbox{otherwise.} 
	\end{array}
	\right. \]

\begin{theorem}
This is the first theorem.
\end{theorem}

\begin{proof}
This is the proof of the first theorem. We show how to write pseudo-code now.
%*** USE PSEUDO-CODE ONLY IF IT IS CLEARER THAN AN ENGLISH DESCRIPTION

Consider a comparison between $x$ and~$y$:
\begin{tabbing}
\hspace*{.25in} \= \hspace*{.25in} \= \hspace*{.25in} \= \hspace*{.25in} \= \hspace*{.25in} \=\kill
\>{\bf if} $x$ or $y$ or both are in $S$ {\bf then } \\
\>\> answer accordingly \\
\>{\bf else} \\
\>\>    Make the element with the larger score (say $x$) win the comparison \\
\>\> {\bf if} $F(x) + F(y) < \frac{n}{t-1}$ {\bf then} \\%
\>\>\> $F(x) \leftarrow F(x) + F(y)$ \\
\>\>\> $F(y) \leftarrow 0$ \\
\>\> {\bf else}  \\
\>\>\> $S \leftarrow S \cup \{ x \} $ \\
\>\>\> $r \leftarrow r+1$ \\
\>\> {\bf endif} \\
\>{\bf endif} 
\end{tabbing}

This concludes the proof.
\end{proof}


\section{Next topic}

Here is a citation, just for fun~\cite{CW87}.

\section*{References}
\beginrefs
\bibentry{CW87}{\sc D.~Coppersmith} and {\sc S.~Winograd}, 
``Matrix multiplication via arithmetic progressions,''
{\it Proceedings of the 19th ACM Symposium on Theory of Computing},
1987, pp.~1--6.
\endrefs

% **** THIS ENDS THE EXAMPLES. DON'T DELETE THE FOLLOWING LINE:
\lecture{2}{January 21}{Dr. Dipti Misra}{CPG,Intro}

\end{document}





